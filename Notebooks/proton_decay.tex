\documentclass[journal, a4paper]{IEEEtran}

\usepackage{graphicx}
%\usepackage{subfigure}
\usepackage{url}

\usepackage{amsmath}    % From the American Mathematical Society
% A popular package that provides many helpful commands
% for dealing with mathematics. Note that the AMSmath
% package sets \interdisplaylinepenalty to 10000 thus
% preventing page breaks from occurring within multiline
% equations. Use:
%\interdisplaylinepenalty=2500
% after loading amsmath to restore such page breaks
% as IEEEtran.cls normally does. amsmath.sty is already
% installed on most LaTeX systems. The latest version
% and documentation can be obtained at:
% http://www.ctan.org/tex-archive/macros/latex/required/amslatex/math/



% Other popular packages for formatting tables and equations include:

%\usepackage{array}
% Frank Mittelbach's and David Carlisle's array.sty which improves the
% LaTeX2e array and tabular environments to provide better appearances and
% additional user controls. array.sty is already installed on most systems.
% The latest version and documentation can be obtained at:
% http://www.ctan.org/tex-archive/macros/latex/required/tools/

% V1.6 of IEEEtran contains the IEEEeqnarray family of commands that can
% be used to generate multiline equations as well as matrices, tables, etc.

% Also of notable interest:
% Scott Pakin's eqparbox package for creating (automatically sized) equal
% width boxes. Available:
% http://www.ctan.org/tex-archive/macros/latex/contrib/supported/eqparbox/

% *** Do not adjust lengths that control margins, column widths, etc. ***
% *** Do not use packages that alter fonts (such as pslatex).         ***
% There should be no need to do such things with IEEEtran.cls V1.6 and later.
\usepackage[
backend=biber,
style=phys,
sortlocale=de_DE,
natbib=true,
url=false, 
doi=true,
eprint=false
]{biblatex}
\addbibresource{NucleonDecay.bib}

% Your document starts here!
\begin{document}
	
	% Define document title and author
	\title{Proton Decay Unidoc}
	\author{Kevin Wierman}
	\maketitle
	
	% Write abstract here
	\begin{abstract}
		Currently, this is empty but will contain the specific language we'll be using in future papers.
	\end{abstract}
	
	% Each section begins with a \section{title} command
	\section{Introduction}
		Conservation of baryon number is “accidental”, i.e does not correspond to any known symmetry: its violation is predicted by almost all GUTs
		
	    TODO:  Need citation here

		Proton (or bounded neutron) decay can occur only as a violation of baryon number, pointing to an indirect evidence for GUTs

    TODO:  Once again, citation
    
    The golden channels:
    \begin{enumerate}
		\item  $p\rightarrow \overline \nu_\tau +K^+$    
		\item $p\rightarrow e^+ + \pi^0$
    \end{enumerate}
	Different GUTs use different decay modes, and LAr may have different sensitivities to these.
	
	% Main Part
	\section{Previous Work}
		\subsection{T2K}

		The summary of results can be found in the collaboration paper~\cite{Collaboration1998}.
		
		A much more detailed paper can be found in B. Viren's PhD thesis\cite{Viren2000}.
		
		Best limits on $\tau_{p\rightarrow e^+ + \pi^0}/B$ set by Super-Kamiokande at $5.4\times10^{33}$ years (90\%C.L.).
		
		This roughly goes as:
		
		\begin{equation}
			\tau/B = \frac{N_0 \Delta t \epsilon}{n_{obs}-n_{bg}}
		\end{equation}
		
		Where $N_0$ is the number of decay centers, $\Delta t$ is the amount of time being measured, and $n_{obs}-n_{bg}$ is the number of observed decays divided by the number of observed bg events.
		Efficiency is measured as $\epsilon$ for the decay mode.
		
		Brett's thesis prefers to use the decay rate or inverse of this
		
		\begin{equation}
		\Gamma = \frac{n}{NT\epsilon}
		\end{equation}
		
		The language here differs a little.
		
		\subsection{$\mu$BooNE}
		
		The previous researcher leading up proton decay studies was Elena Gramellini. 
		She frequently references Ed Kearns who has a ton of informative plots.
		
		This will a more comprehensive search of the microboone docdb.
	
	\section{Referencing the Software}
		This section will be used to develop the language we'll be using to properly reference the software in LArSoft\cite{LArSoft2016}.
		
		
	\section{The Goal}
	
	This is done following Brett Virren's thesis.
	
	
	The confidence level for the limit on the decay rate, $\Gamma_{lim}$ can be calculated as,
	
	\begin{equation}
		CL = A\int_0^{\Gamma_{lim}}d\Gamma\int_0^\infty db\int_0^\infty db_{MC}\int_0^{\infty}d\lambda\int_0^1 d\epsilon I(...)
	\end{equation}
	
	Where,
	
	\begin{equation}
		I(...) = \frac{b_{MC}^{n_b}(\Gamma\lambda \epsilon+b)^n}{b}
		e^{-b_{MC}-\frac{(b_{MC}-bC)^2}{2\sigma_b^2}-\frac{(\lambda-\lambda_0)^2}{2\sigma^2_\lambda }-(\Gamma\lambda\epsilon+b)}
	\end{equation}
	Where
	\begin{itemize}
		\item $b_{MC}$, the true mean MC background rate.
		\item $\Gamma$, the true decay rate of interest (integration parameter)
		\item $\lambda = NT$, the true exposure
		\item $b$, the true background rate (decay parameter)
		\item $n$, the number of events observed
		\item $\lambda_0$, the estimated exposure
		\item $\sigma_\lambda$, the uncertainty in the estimated exposure
		\item $C$, the ratio of MC events to data events.
		\item $n_b$, the number of MC events passing the the proton decay selection criteria
		\item $\epsilon$ The efficiency of detecting the decay products of this mode.
	\end{itemize}
	
	Specifically, the parameters that need to be measured are:
	\begin{itemize}
		\item $n$, the number of events observed
		\item $n_b$, the number of MC events passing the the proton decay selection criteria
		\item $\sigma_b$ The estimated uncertainty in the MC background rate
		\item $\lambda_0$, the estimated exposure
		\item $\sigma_\lambda$, the uncertainty in the estimated exposure
		\item $C$, the ratio of MC events to data events.$\equiv n/(n_b+n_o)$
		\item $\epsilon_0$ Estimated efficiency
		\item $\sigma_\epsilon$ Uncertainty in estimated efficiency
	\end{itemize}	

	The constant of proportionality, $A$ is found by setting $CL=1$ and $\Gamma_{lim}$ to $\infty$.
	
	The efficiency term missing from his paper is:
	
	\begin{equation}
		P(\epsilon| \vec I) = e^{-(\epsilon-\epsilon_0)^2/2\sigma_{\epsilon}^2}
	\end{equation}
	
	which should be in the integrand.
	The estimated efficiency for each mode is found by dividing the number of proton decay MC events which pass all cuts by the number which pass the event sample cuts.
	
	For a given CL, evaluate the integral numerically using a minimization routine to obtain $\Gamma_{lim}$
	
	\section{General Method}
	
	This part follows Elena's reasoning
	
	\begin{enumerate}
		\item Pick a decay mode (topology).
		\item Evaluate the background processes that can contribute to identifying the decay products.
		\item Define cuts for a given process which maximize the efficiency.
		\item For these cuts, obtain:
			\begin{itemize}
				\item The MC background rate and uncertainty
				\item The estimated exposure (fiducial cuts?)
				\item The number of events in the data
				\item The number of background events passing selection criteria in MC
				\item the ratio of all MC events to all data events which pass selection criteria
				\item The estimated efficiency
			\end{itemize} 
		\item Evaluate integral
	\end{enumerate}
	
	\section{Backgrounds}

		The big aspect behind this will be background identification.
		
		The cuts that were mentioned are:
		
		\begin{itemize}
			\item production PID
			\item No additional vertex activity
			\item Proximity
			\item Planarity
			\item Total Energy deposited
			\item Total Net Momentum
			\item Fiducialization
		\end{itemize}
		
		Elena's work was based around examining the MCC5 cosmics generation data sets.
		A year ago, she was working on cuts.	
	
	\section{Conclusion}
	This section summarizes the paper.
	\printbibliography 

\end{document}